\section{Structure and Organization}

The NumNN.jl  package\footnote{\url{https://github.com/MohHizzani/NumNN.jl}} (library) was built on the concept of Rapid Prototyping to enable fast experimentation. \emph{Begin able to go from idea to result with the least possible delay is key to good research} \cite{Keras}. There are so many deeplearning frameworks out on the internet \cite{Abadi2016,Collet2015,Jia2014,Paszke2017,PyTorch2019}, some are widely spread and efficient in production and research as long as you are using the IEEE-754 standards. However, most of these libraries run on Python which has not the native support of newly added primitive types nor the efficiency of Julia in multiple dispatch (i.e. any new defined number systems needs an internal modification to the library itself to be capable of handling it). Even packages of Julia like Flux.jl \cite{Flux.jl-2018,Innes2018} or Knet \cite{Yuret2016k} neither of them fully support newly added types, and do not provide the flexibility of different types for different parameters or to deliver output with certain desired data types. Thus, NumNN.jl provides a fully support for a total in one type computation, the ability to choose the type of each group of parameters and the capabilities of delivering output of a chosen type.

\subsection{Fast Implementation and Deployment}

NumNN.jl provides the fastest possible way to implement a neural network model, train and use it for prediction and testing. Installing the package and using it can be done from a Julia REPL as follows:

\begin{listing}[H]
	\begin{minted}{julia}
	julia> ] add NumNN
	julia> using NumNN
	\end{minted}
	\caption{Adding NumNN.jl and import it}\label{addimport}
\end{listing}

Defining layers sequence can be easily done either like this:

\begin{listing}[H]
	\begin{minted}{julia}
	X_Input, X_Ouput = chain(X_train,[Flatten(),FCLayer(120,:relu),FCLayer(10,:softmax)])
	\end{minted}
	\caption{Chained Layers with no side branch(es)}\label{chain}
\end{listing}


where \mintinline[fontsize=\footnotesize]{julia}{X_train} is the train array or the size of train array. Or in case there are some side branches it can be easily done as the example of Inception Net \cite{Szegedy2016} (Listing~\ref{chain}). 

%A group of examples can be found in Appendices.

\subsection{Structure and Architecture}\label{subsec:saa}

The structure of NumNN.jl is simple enough to comprehend in few minutes and to start to amend in case any new element is needed to be added. \figurename~\ref{fig:layerstruct} shows the Layers hierarchy in NumNN.jl, this structure was built to facilitate the use and the future development of NumNN.jl.

The convention of data shape to be used is (data dimension(s), channels, batch size). For instance, when using MNIST \cite{LeCun1998,LeCun1998a} data set the shape of training data will be (28,28,1,60000), where the first two dimensions are the images' dimensions the third one is number of channels and the last integer expresses the batch size.

\begin{figure}[!ht]
	\centering
	%\begin{subfigure}{0.95\textwidth}
		%\begin{enumerate}
%\qtreecenterfalse
%\item \Tree [.Layer  FCLayer Activation BatchNorm Flatten Input ]

%\item \Tree [.Layer [.MILayer AddLayer ConcatLayer ]]


%\item \Tree [.Layer [.PaddableLayer
%[.ConvLayer Conv1D Conv2D Conv3D ]
%[.PoolLayer
%[.AveragePoolLayer AveragePool1D AveragePool2D AveragePool3d ]
%[.MaxPoolLayer MaxPool1D MaxPool2d MaxPool3d ]]]]
%\end{enumerate}
%\documentclass{article}
%\usepackage{tikz}
%\usetikzlibrary{mindmap}
%\begin{document}


\begin{tikzpicture}[mindmap,
every node/.style={concept,
	execute at begin node=\hskip0pt}, concept color=orange!40,% concept fontsize=10pt,
grow cyclic, %text width=3cm, align=flush center,
level 1/.append style={level distance=4cm,sibling angle=51
},
level 2/.append style={level distance=2.7cm,sibling angle=70
},
level 3/.append style={level distance=2.3cm, sibling angle=70
},
level 4/.append style={level distance=1.8cm,sibling angle=60
}
%scale=0.9, transform canvas={scale=0.9}
]
\node[scale=0.8]{\LARGE Layer}
child [concept color=green!45] { node {FCLayer}}
child [concept color=green!45] { node {Activation}}
child [concept color=green!45] { node {BatchNorm}}
child [concept color=green!45] { node {Flatten}}
child [concept color=green!45] { node {Input}}
child [concept color=blue!50, scale=1.2] { node {PaddableLayer}
																	child [sibling angle=90] { node {PoolLayer}
																	child [sibling angle=80] { node [scale=1.5] {MaxPoolLayer}
																		child { node [scale=1.5] {MaxPool1D}}
																		child { node [scale=1.5] {MaxPool2D}}
																		child { node [scale=1.5] {MaxPool3D}}
																	}
																	child { node [scale=1.4] {AveragePoolLayer}
																		child { node [scale=1.4] {AveragePool1D}}
																		child { node [scale=1.4] {AveragePool2D}}
																		child { node [scale=1.4] {AveragePool3d}}
																	}
																}
																child { node {ConvLayer}
																	child [level distance=2cm] { node [scale=1.3] {Conv1D}}
																	child [level distance=2cm] { node [scale=1.3] {Conv2D}}
																	child [level distance=2cm] { node [scale=1.3] {Conv3D}}
															}}
child [concept color=red!45] { node {MILayer}
	child { node {AddLayer}}
	child { node {ConcatLayer}}
			};
\end{tikzpicture}

%\end{document}
	%\end{subfigure}
	\caption{Layer Architecture in NumNN.jl, the useable layers are the leaves the other nodes are \mintinline[fontsize=\footnotesize]{julia}{abstract type}s.}\label{fig:layerstruct}
\end{figure}

An essential data type (\mintinline[fontsize=\footnotesize]{julia}{Model}) holds the main pointers to layers structure and parameters to use. Instantiating a (\mintinline[fontsize=\footnotesize]{julia}{Model}) will also invoke the initialization of the layers scaling and bias parameters and can be done as presented in Listing~\ref{modelinit}.

\begin{listing}[H]
	\begin{minted}{julia}
	model = Model(X_train, Y_train, X_Input, X_Output, 0.001; optimizer=:adam, loss=:categoricalCrossentropy)
	\end{minted}
	\caption{Model initialization, \mintinline{julia}{X_train, Y_train} are training data and labels, while \mintinline{Julia}{X_Input, X_Ouput} are the input and output layers. The value of \mintinline{julia}{0.001} represent the learning rate of this model, where the key-word \mintinline{julia}{optimizer} define the optimizer to use during training, and \mintinline{julia}{loss} defines the loss function.}\label{modelinit}
\end{listing}

\begin{listing}[!h]
	\begin{minted}{julia}
	X_Input = Input(X_train)
	Xc = [
	Conv2D(3, (3,3), padding=:same)(X_Input),
	Conv2D(4, (5,5), padding=:same)(X_Input),
	Conv2D(10, (1,1), padding=:same)(X_Input),
	MaxPool2D((2,2); padding=:same)(X_Input),
	AveragePool2D((3,3); padding=:same)(X_Input),
	]
	X = ConcatLayer()(Xc)
	X = BatchNorm(dim=3)(X) #to normalize across the channels
	X = Activation(:relu)(X)
	X = MaxPool2D((2,2))(X);
	Xc = [
	Conv2D(6, (3,3), padding=:same)(X),
	Conv2D(8, (5,5), padding=:same)(X),
	Conv2D(10, (1,1), padding=:same)(X),
	MaxPool2D((2,2); padding=:same)(X),
	AveragePool2D((3,3); padding=:same)(X),
	]
	X = ConcatLayer()(Xc)
	X = BatchNorm(dim=3)(X) #to normalize across the channels
	X = Activation(:relu)(X)
	X = AveragePool2D((3,3))(X);
	X = Flatten()(X)
	X_Output = FCLayer(10, :softmax)(X);
	\end{minted}
	\caption{InceptionNet Example where this layer architecture has many side branches.}\label{chain}
\end{listing}

Activation functions are defined under the \mintinline[fontsize=\footnotesize]{julia}{abstract type actFun}. NumNN.jl only provided few activation functions (sigmoid ($\sigma$), \mintinline{julia}{tanh}, \mintinline[fontsize=\footnotesize]{julia}{softmax} and \mintinline[fontsize=\footnotesize]{julia}{noAct}). Using any other activation function can be defined as a subtype of \mintinline[fontsize=\footnotesize]{julia}{actFun} and then as a normal function with another function for its derivative. For instance, if you want to add $\sin$ as an activation function all what you need is expressed in Listing~\ref{newact}.

\begin{listing}[!ht]
	\begin{minted}{julia}
	abstract type sin <: actFun end
	
	sin(x) = base.sin.(x) #note the use of base. to tell julia that we need the default sin at this step not the type defined because it was overwritten
	dsin(x) = base.cos.(x)
	\end{minted}
	\caption{Example of defining a new activation function to NumNN.jl}\label{newact}
\end{listing}

NumNN.jl has loss functions under the \mintinline{julia}{abstract type lossFun}. The two main loss functions defined are the \mintinline{julia}{binaryCrossentropy} and \mintinline{julia}{categoricalCrossentropy}. Also, other loss functions can be easily defined, more functions will be available in the upcoming versions of NumNN.jl.

%\subsection{Sequence of Processes}
%The easiest way of explaining NumNN.jl is by showing a complete example. 

\subsection{Some Benchmarks and Testing}
The most convenient way to demonstrate the effectiveness of NumNN.jl is by getting hands dirty and showing some examples accompanied with comparison beside another library. Keras on the top of TensorFlow was one of the main influencers upon the syntax of NumNN.jl. Thus, some benchmarks were run on both NumNN and TensorFlow. Nonetheless, other benchmarks could not be run on TensorFlow due to lack support for new data types, therefore, they were run only on NumNN.jl.

The data used for train/test was Fashion MNIST \cite{Xiao2017}. Intel(R) Xeon(R) Gold 6140 CPU @ 2.30GHz was used for all tests, and no GPU were used (NumNN does not support GPU and other back-end so far). Table~\ref{tab:bench} shows the results of all test\footnote{All benchmarks are available at \url{https://github.com/MohHizzani/NumNN.jl/tree/master/examples}}, and where bank is present that means test was not perform under this framework because the data type is not supported\footnote{\label{batchnorm}Benchmark 2 \& 3 has Batch Normalization layer where Float64 is not supported for this layer in TensorFlow}.

\begin{table}[!ht]
	\centering
	\renewcommand{\arraystretch}{1.1}
	\caption{Various benchmarks for NumNN.jl and TensorFlow}\label{tab:bench}
	\begin{tabular}{|c | c | c || c | c |}
		\hline
		Benchmark & DType & Measurements & NumNN.jl & Tensorflow\\\hline\hline
		%%%%%%%%%% Benchmark 1
		\multirow{14}{*}{Benchmark 1} & \multirow{5}{*}{Float32} & Train Accuracy & 0.8902 & 0.9162\\
		& & Test Accuracy & 0.8650 & 0.8851\\
		& & Train Cost & 0.3000 & 0.2270\\
		& & Test Cost & 0.3728 & 0.3444\\
		& & Training Time / Sample & $84\mu$s  & $177\mu$s\\\cline{2-5}
		& \multirow{5}{*}{Float64} & Train Accuracy & 0.8812 & 0.9191 \\
		& & Test Accuracy & 0.8703 & 0.8848 \\
		& & Train Cost & 0.3314 & 0.2163 \\
		& & Test Cost & 0.3845 & 0.3311 \\
		& & Training Time / Sample & $93\mu$s & $100\mu$s\\\cline{2-5}
		& \multirow{4}{*}{Posit\{16,1\}} & Train Accuracy & 0.8716 & \rule{5em}{1pt} \\
		& & Test Accuracy & 0.8510 & \rule{5em}{1pt} \\
		& & Train Cost & 0.3520 & \rule{5em}{1pt} \\
		& & Test Cost & 0.4088 & \rule{5em}{1pt} \\\hline
		%%%%%%%%%% Benchmark 2
		\multirow{14}{*}{Benchmark 2\footref{batchnorm}} & \multirow{5}{*}{Float32} & Train Accuracy & 0.8980 & 0.9160\\
		& & Test Accuracy & 0.8786 & 0.8929\\
		& & Train Cost & 0.2760 & 0.0.2346\\
		& & Test Cost & 0.3386 & 0.2955\\
		& & Training Time / Sample & $1.31m$s & $8m$s \\\cline{2-5}
		& \multirow{5}{*}{Float64} & Train Accuracy & 0.9020 & \rule{5em}{1pt} \\
		& & Test Accuracy & 0.8857 & \rule{5em}{1pt} \\
		& & Train Cost & 0.2750 & \rule{5em}{1pt} \\
		& & Test Cost & 0.3420 & \rule{5em}{1pt} \\
		& & Training Time / Sample & $1.34m$s & \rule{5em}{1pt} \\\cline{2-5}
		& \multirow{4}{*}{Posit\{16,1\}} & Train Accuracy & 0.9114 & \rule{5em}{1pt} \\
		& & Test Accuracy & 0.8949 & \rule{5em}{1pt} \\
		& & Train Cost & 0.2453 & \rule{5em}{1pt} \\
		& & Test Cost & 0.3045 & \rule{5em}{1pt} \\\hline
		%%%%%%%%%% Benchmark 3
		\multirow{14}{*}{Benchmark 3\footref{batchnorm}} & \multirow{5}{*}{Float32} & Train Accuracy & 0.9136 & 0.9072\\
		& & Test Accuracy & 0.8953 & 0.8899\\
		& & Train Cost & 0.2362 & 0.2527\\
		& & Test Cost & 0.2962 & 0.3050\\
		& & Training Time / Sample & $3.09m$s & $14m$s\\\cline{2-5}
		& \multirow{5}{*}{Float64} & Train Accuracy & 0.9103 & \rule{5em}{1pt} \\
		& & Test Accuracy & 0.8926 & \rule{5em}{1pt} \\
		& & Train Cost & 0.2480 & \rule{5em}{1pt} \\
		& & Test Cost & 0.3067 & \rule{5em}{1pt} \\
		& & Training Time / Sample & $3.3m$s & \rule{5em}{1pt} \\\cline{2-5}
		& \multirow{4}{*}{Posit\{16,1\}} & Train Accuracy & 0.9127 & \rule{5em}{1pt} \\
		& & Test Accuracy & 0.8954 & \rule{5em}{1pt} \\
		& & Train Cost & 0.2412 & \rule{5em}{1pt} \\
		& & Test Cost & 0.2999 & \rule{5em}{1pt} \\\hline		
	\end{tabular}
\end{table}


