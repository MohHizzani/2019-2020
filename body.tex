\section{NumNN.jl}

This package (library) was build on the concept of rapid prototyping to enable fast experimentation. \emph{Begin able to go from idea to result with the least possible delay is key to good research}\cite{Keras}.

\subsection{Fast Implementation and Deployment}

NumNN.jl provides the fastest possible way of implement a neural network model, train and use it for prediction and testing. Installing the package and using it can be done from a Julia REPL as follows:

\begin{listing}[H]
	\begin{minted}[frame=single,fontsize=\footnotesize ,breaklines]{julia}
	julia> ] add NumNN
	julia> using NumNN
	\end{minted}
	\caption{Adding NumNN.jl and import it}\label{addimport}
\end{listing}

Defining layers sequence can be easily done either like this:

\begin{listing}[H]
	\begin{minted}[frame=single,fontsize=\footnotesize ,breaklines]{julia}
	X_Input, X_Ouput = chain(X_train,[Flatten(),FCLayer(120,:relu),FCLayer(10,:softmax)])
	\end{minted}
	\caption{Chained Layers with no side branch(es)}\label{chain}
\end{listing}


where \mintinline[fontsize=\footnotesize]{julia}{X_train} is the train array or the size of train array. Or in case there is some side branches it can be easily done as follow:

\begin{listing}[H]
	\begin{minted}[frame=single,fontsize=\footnotesize ,breaklines]{julia}
	X_Input = Input(X_train)
	Xc = [
	Conv2D(3, (3,3), padding=:same)(X_Input),
	Conv2D(4, (5,5), padding=:same)(X_Input),
	Conv2D(10, (1,1), padding=:same)(X_Input),
	MaxPool2D((2,2); padding=:same)(X_Input),
	AveragePool2D((3,3); padding=:same)(X_Input),
	]
	X = ConcatLayer()(Xc)
	X = BatchNorm(dim=3)(X) #to normalize across the channels
	X = Activation(:relu)(X)
	X = MaxPool2D((2,2))(X);
	Xc = [
	Conv2D(6, (3,3), padding=:same)(X),
	Conv2D(8, (5,5), padding=:same)(X),
	Conv2D(10, (1,1), padding=:same)(X),
	MaxPool2D((2,2); padding=:same)(X),
	AveragePool2D((3,3); padding=:same)(X),
	]
	X = ConcatLayer()(Xc)
	X = BatchNorm(dim=3)(X) #to normalize across the channels
	X = Activation(:relu)(X)
	X = AveragePool2D((3,3))(X);
	X = Flatten()(X)
	X_Output = FCLayer(10, :softmax)(X);
	\end{minted}
	\caption{InceptionNet Example}\label{chain}
\end{listing}

A group of examples can be found in Appendices.

\subsection{Structure and Architecture}

The structure of NumNN.jl is simple enough to comprehend in few minutes and to start amend in case any new elements is needed to be added. Figure \ref{fig:layerstruct} shows the Layers hierarchy in NumNN.jl, this structure was built to facilitate the use and the future development of NumNN.jl.

\begin{figure}[H]
	\centering
	%\begin{subfigure}{0.95\textwidth}
		%\begin{enumerate}
%\qtreecenterfalse
%\item \Tree [.Layer  FCLayer Activation BatchNorm Flatten Input ]

%\item \Tree [.Layer [.MILayer AddLayer ConcatLayer ]]


%\item \Tree [.Layer [.PaddableLayer
%[.ConvLayer Conv1D Conv2D Conv3D ]
%[.PoolLayer
%[.AveragePoolLayer AveragePool1D AveragePool2D AveragePool3d ]
%[.MaxPoolLayer MaxPool1D MaxPool2d MaxPool3d ]]]]
%\end{enumerate}
%\documentclass{article}
%\usepackage{tikz}
%\usetikzlibrary{mindmap}
%\begin{document}


\begin{tikzpicture}[mindmap,
every node/.style={concept,
	execute at begin node=\hskip0pt}, concept color=orange!40,% concept fontsize=10pt,
grow cyclic, %text width=3cm, align=flush center,
level 1/.append style={level distance=4cm,sibling angle=51
},
level 2/.append style={level distance=2.7cm,sibling angle=70
},
level 3/.append style={level distance=2.3cm, sibling angle=70
},
level 4/.append style={level distance=1.8cm,sibling angle=60
}
%scale=0.9, transform canvas={scale=0.9}
]
\node[scale=0.8]{\LARGE Layer}
child [concept color=green!45] { node {FCLayer}}
child [concept color=green!45] { node {Activation}}
child [concept color=green!45] { node {BatchNorm}}
child [concept color=green!45] { node {Flatten}}
child [concept color=green!45] { node {Input}}
child [concept color=blue!50, scale=1.2] { node {PaddableLayer}
																	child [sibling angle=90] { node {PoolLayer}
																	child [sibling angle=80] { node [scale=1.5] {MaxPoolLayer}
																		child { node [scale=1.5] {MaxPool1D}}
																		child { node [scale=1.5] {MaxPool2D}}
																		child { node [scale=1.5] {MaxPool3D}}
																	}
																	child { node [scale=1.4] {AveragePoolLayer}
																		child { node [scale=1.4] {AveragePool1D}}
																		child { node [scale=1.4] {AveragePool2D}}
																		child { node [scale=1.4] {AveragePool3d}}
																	}
																}
																child { node {ConvLayer}
																	child [level distance=2cm] { node [scale=1.3] {Conv1D}}
																	child [level distance=2cm] { node [scale=1.3] {Conv2D}}
																	child [level distance=2cm] { node [scale=1.3] {Conv3D}}
															}}
child [concept color=red!45] { node {MILayer}
	child { node {AddLayer}}
	child { node {ConcatLayer}}
			};
\end{tikzpicture}

%\end{document}
	%\end{subfigure}
	\caption{Layer Architecture in NumNN.jl}\label{fig:layerstruct}
\end{figure}

